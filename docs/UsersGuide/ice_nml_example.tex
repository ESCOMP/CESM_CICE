%=======================================================================
% CVS: $Id: ice_nml_example.tex 5 2005-12-12 17:41:05Z mvr $
% CVS: $Source$
% CVS: $Name$
%=======================================================================

This section shows several examples of namelists from the coupled and uncoupled
ice models.  These examples are taken directly from {\bf csim.buildnml\_prestage.csh}
for the coupled model and from {\bf csim.setup.csh} for the uncoupled model.  
Most of the variables in the namelist are determined from environment variables
set elsewhere in the scripts. Since the namelists from the coupled model are
"resolved" by the scripts, meaning that the values of most of the shell script
variables are put directly into the namelist, examples are shown for the most
commonly used configurations.  Variables that are commonly changed directly in
the namelist are the timestep {\tt dt} and the number of subcycles per timestep
in the ice dynamics {\tt ndte}. 

\subsubsection {Example 1: CCSM Fully Coupled}
\label{example1_nml}

The following example is the namelist used for CCSM fully coupled, or
the B configuration.  The variables that are still set to shell script
variables have been set at the top of {\bf csim.buildnml\_prestage.csh}
or in other scripts.  A completely resolved version of the namelist will
be written to {\bf ice\_in} in the executable directory.

\begin{verbatim}
cat << EOF >! ice_in
  &ice_nml
    runid         = 'TER.01a.T85_gx1v3.B.bluesky.105608'
  , runtype       = '$runtype'
  , istep0        = 0
  , dt            = 3600.0
  , ndte          = 120
  , ndyn_dt       = 1
  , npt           = 99999
  , diagfreq      = 24
  , histfreq      = 'm'
  , dumpfreq      = 'y'
  , dumpfreq_n    = 1
  , hist_avg      = .true.
  , restart       = $restart
  , print_points  = .false.
  , kitd          = 1
  , kdyn          = 1
  , kstrength     = 1
  , evp_damping   = .false.
  , snow_into_ocn = .true.
  , advection     = 'remap'
  , grid_type     = 'displaced_pole'
  , grid_file     = 'data.domain.grid'
  , kmt_file      = 'data.domain.kmt'
  , incond_dir    = '$runinidir/'
  , incond_file   = '$CASE.csim.i.'
  , restart_dir   = '$runrstdir/'
  , dump_file     = '$CASE.csim.r.'
  , history_dir   = '$runhstdir/'
  , history_file  = '$CASE.csim.h'
  , albicev       = 0.73
  , albicei       = 0.33
  , albsnowv      = 0.96
  , albsnowi      = 0.68
  , no_ice_ic     = $no_ice_ic
  , oceanmixed_ice             = .false.
  , oceanmixed_ice_file        = 'oceanmixed_ice.nc'
  , oceanmixed_ice_sst_init    = .false.
  , prntdiag_oceanmixed        = .false.
  , pointer_file  = 'rpointer.ice'
/
\end{verbatim}

\subsubsection {Example 2: Coupled Ice Only Model}
\label{example2_nml}

This example is the M configuration.  It is CSIM with the latm data
atmosphere model, data land model, and the ocean mixed layer model
within the ice model all communicating through the coupler.  The 
following modifications will be made to the namelist when the resolved
scrips are created for the M configuration.  See the 
\begin{htmlonly}
  \htmladdnormallink{CCSM3 User's Guide}{http://www.ccsm.ucar.edu/models/ccsm3.0/ccsm}
\end{htmlonly}
\begin{latexonly}
  CCSM3 User's Guide (http://www.ccsm.ucar.edu/models/ccsm3.0/ccsm)
\end{latexonly}
on how to create scripts for the M configuration.

\begin{verbatim}
    runid         = 'TER.01a.T62_gx1v3.M.bluesky.095234'
  , oceanmixed_ice             = .true.
  , oceanmixed_ice_sst_init    = $oml_ice_sst_init
\end{verbatim}

\subsubsection {Example 3: Uncoupled Ice Model}
\label{example3_nml}

This example is the namelist from the uncoupled ice model that resides
in the file {\bf csim.setup.csh}.  {\tt npt}  has been modified, since
it determines the length of the uncoupled run.  The snow and ice albedos
used by CCSM are not set in the name list. The default values set in
{\bf ice\_init.F} are used.  

\begin{verbatim}
cat << EOF >! ice_in
  &ice_nml
    runid        = '$CASE $CASESTR'
  , runtype      = '$RUNTYPE'
  , istep0       = 0
  , dt           = 3600.0
  , ndyn_dt      = 1
  , ndte         = 120
  , npt          = 240
  , diagfreq     = 24
  , histfreq     = 'm'
  , dumpfreq     = 'd'
  , dumpfreq_n   = 5
  , hist_avg     = .true.
  , restart      = $RESTART
  , print_points = .false.
  , kitd         = 1
  , kdyn         = 1
  , kstrength    = 1
  , evp_damping  = .false.
  , snow_into_ocn = .true.
  , advection    = 'remap'
  , grid_type    = 'displaced_pole'
  , grid_file    = 'data.domain.grid'
  , kmt_file     = 'data.domain.kmt'
  , incond_dir   = '$INIDIR/'
  , incond_file  = '$CASE.csim.i.'
  , restart_dir  = '$RSTDIR/'
  , dump_file    = '$CASE.csim.r.'
  , history_dir  = '$HSTDIR/'
  , history_file = '$CASE.csim.h'
  , pointer_file  = '$CSIMDIR/rpointer.ice'
  , oceanmixed_ice             = $OCEANMIXED_ICE
  , oceanmixed_ice_file        = 'oceanmixed_ice.nc'
  , oceanmixed_ice_sst_init    = $OML_ICE_SST_INIT
  , prntdiag_oceanmixed        = .false.
/
\end{verbatim}

\subsubsection {Example 4: Uncoupled Ice Model with Atmospheric Forcing}
\label{example4_nml}

To run the uncoupled ice model with atmospheric forcing, the following
options should be added to the above namelist:

\begin{verbatim}
  , ycycle       = 1
  , year_init    = 1997
  , fyear_init   = 1997
  , atm_data_dir = '/ptmp/$LOGNAME/csim_forcing/atm/gx3v5/'

\end{verbatim}


\subsubsection {Example 5: History File Namelist}
\label{example5_nml}

The second namelist controls what variables are written to the history file.
By default, all files are written to the history file.  Variables that
are not output are set in the namelist {\tt icefields\_nml}.
Some of the following fields are not written to the history file since
they can be retrieved from the ocean history files.  The melt and freeze
onset fields are not used, since the information they contain may not
be correct if the model is restarted mid-year.  The ice areas and volumes
for categories six through ten are not used, since the default thickness
distribution consists of five ice categories.

\begin{verbatim}
  &icefields_nml
    f_sst       = .false.
  , f_sss       = .false.
  , f_uocn      = .false.
  , f_vocn      = .false.
  , f_frzmlt    = .false.
  , f_strtltx   = .false.
  , f_strtlty   = .false.
  , f_mlt_onset = .false.
  , f_frz_onset = .false.
  , f_aice6     = .false.
  , f_aice7     = .false.
  , f_aice8     = .false.
  , f_aice9     = .false.
  , f_aice10    = .false.
  , f_vice1     = .false.
  , f_vice2     = .false.
  , f_vice3     = .false.
  , f_vice4     = .false.
  , f_vice5     = .false.
  , f_vice6     = .false.
  , f_vice7     = .false.
  , f_vice8     = .false.
  , f_vice9     = .false.
  , f_vice10    = .false.
  /
\end{verbatim}
