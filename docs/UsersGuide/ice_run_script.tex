%=======================================================================
% CVS: $Id: ice_run_script.tex 5 2005-12-12 17:41:05Z mvr $
% CVS: $Source$
% CVS: $Name$
%=======================================================================
% \subsection{Uncoupled Run Script}

The run script for the uncoupled model is called {\bf csim\_run} and is
located in {\bf /ccsm3/models/ice/csim4/src}.  Its purpose is to coordinate setting
the batch system options, the environment variables, executing the CSIM
setup script, setting up the stdout file, and submitting the model to run.

At the top of the run script, the settings for the IBM SP and the SGI Origin
2000 batch queue environments are set.  These commands are machine and
site dependent.  Following this, the variables for the run environment
are defined.  These variables are listed in Table \ref{table:main_environ_var}.

{\tt \$CASE} is a character string that identifies a particular model run.
It can be up to 16 characters long, but it is best kept short, since
it is used as part of the restart, history, and initial filenames.
{\tt \$CASESTR} is a longer string that describes a model case.

{\tt \$RUNTYPE} is a character string that specifies the state in which
the model is to begin a run.  {\tt startup} and {\tt continue} are the supported
run types. A startup run can be initialized by reading input from a file
or from initial conditions set within the ice model.  This option is
controlled by the environment variable {\tt \$RESTART} in the setup script
(see Section \ref{model_generated} ). Continue runs are described in
Section \ref{continue_runs}.

{\tt \$NCAT} is an integer that sets the number of ice
thickness categories.  The default value is 5 categories.  If you are 
considering changing {\tt \$NCAT} to values other than 3 or 10, read
Section \ref{ncat}.  This is an involved process that deserves
its own section.

\subsubsection{Using the Ocean Mixed Layer Model within CSIM}

{\tt \$OCEANMIXED\_ICE} is a logical variable, if {\tt .true.}, is used
to implement the slab ocean mixed layer model in the ice model.
It is a simple model that is forced using output from a POP ocean
simulation.  More details on the physics of the ocean mixed layer model
can be found in the Physics Documentation.  It can be run
with the gx3v5 or the gx1v3 grid.  To use the mixed layer model, set

  \begin{verbatim}
    setenv \$OCEANMIXED\_ICE .true.
  \end{verbatim}
{\noindent in {\bf csim\_run}.}

\subsubsection{Changing Grid Resolution}

{\tt \$GRID} is a character string used to specify the horizontal grid.
Presently, two resolutions are supported for the ice model: gx3v5 and
gx1v3. In both of these grids, the North Pole has been displaced into 
Greenland.  gx3v5 is the coarser grid, with longitudinal resolution of
3.6 degrees. The latitudinal resolution varies, with a resolution of 
approximately 0.9 degrees near the equator. gx1v3 is the finer
resolution grid, with a longitudinal resolution of approximately one
degree. Its latitudinal resolution is also variable, with a resolution
of approximately 0.3 degrees near the equator. 

\subsubsection{Changing the Number of Processors}

{\tt \$NX} and {\tt \$NY} are the number of processors used by the
ice model for internal parallelization.
Currently, {\tt \$NX} and {\tt \$NY} MUST divide evenly into the grid dimensions.
There are checks for this in the setup script and in the ice source
code; the model will stop if these criteria are not met. {\tt \$NLAT} and
{\tt \$NLON} are used for this purpose.  For load balancing purposes, {\tt \$NY}
should be $<$=2.  If it is greater than this, the processors assigned
subdomains near the equator will not be doing much work.

For the gx3v5 grid, the ice model is typically run on 8 tasks, with {\tt NX}=4,
{\tt NY}=2. Running the ice model with {\tt NX}=8 and {\tt NY}=1 tasks on the
gx3v5 grid wil result in an error, since 8 does not divide evenly into the 100
longitude points.  When this happens, the model will stop with an error message
written to the log file.

If you are submitting the model to a batch queue with the number of processors
modified from the default, you will also have to modify the batch queue
environment information at the top of the script.  The default setting for the IBM is:

\begin{verbatim}
  # @ total_tasks = 8
  # @ node = 1
\end{verbatim}

These two lines request a total of eight MPI processes, on one 8-way node.
For the NCAR SGI, the default setting is:

\begin{verbatim}
  # QSUB -l mpp_p=8           # request 8 processors
\end{verbatim}

and for other SGI's it may be

\begin{verbatim}
  # BSUB -n 8   .
\end{verbatim}




