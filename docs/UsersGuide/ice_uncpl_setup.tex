%=======================================================================
% CVS: $Id: ice_uncpl_setup.tex 5 2005-12-12 17:41:05Z mvr $
% CVS: $Source$
% CVS: $Name$
%=======================================================================

%\subsection{Uncoupled Setup Script}

The purpose of the setup script, {\bf csim.setup.csh}, is to build an
executable version of the ice model, document what source code and data
files are being used in the {\bf ice.log.\$LID} file, and gather or create
any necessary input data files. {\tt \$LID} is a time stamp set in the run script.
The environment variables set locally in the ice setup script are listed
in Table \ref{table:environ_var}.

\begin{Ventry}{NOTE:}
\item[NOTE]
The variables shown in Table \ref{table:environ_var} will rarely have to
be modified by the user, since they depend on variables set in the run
script.  The most common changes made in the script file will be to the
namelist discussed in Section \ref{namelist}.

\end{Ventry}

\begin{table}
  \begin{center}
  \caption{Environment Variables Set in the Run Script ({\bf csim\_run})}
  \label{table:main_environ_var}
  \begin{tabular}{p{3.5cm}p{9cm}} \hline

  Variable    & Description    \\ \hline \hline
   {\tt CASE}     & case name   \\
   {\tt CASESTR}  & short descriptive text string, used in history files  \\
   {\tt OCEANMIXED\_ICE}  & logical variable used to implement ocean mixed layer model \\
   {\tt ICE\_GRID}& ice model grid (gx3v5 or gx1v3)            \\
   {\tt RUNTYPE}  & run type (startup or continue)\\
   {\tt NCAT}     & number of thickness categories in the ice thickness distribution \\
   {\tt NX}       & number of processors assigned in the x direction,
              used for MPI grid decomposition  \\
   {\tt NY}       & number of processors assigned in the y direction,
              used for MPI grid decomposition  \\
   {\tt BINTYPE}  & Set to MPI for internal parallelization, set to single for non-MPI runs  \\
   {\tt CSIMDIR}  & source code base directory  \\
   {\tt SHRCODE}  & share code directory  \\
   {\tt CSIMDATA} & input data base directory   \\
   {\tt CBLD}     & makefile and Macros directory  \\
   {\tt EXEROOT}  & Run model, mv data, output put here  \\
   {\tt LID}      & timestamp for file ID string  \\
   {\tt OBJDIR}  & ice model code is built here  \\
  \hline
  \end{tabular}
  \end{center}
\end{table}

{\tt \$HSTDIR}, {\tt \$RSTDIR} and {\tt \$INIDIR} are the directories in
{\tt \$EXEROOT} where the history, restart, and initial condition
files are output, respectively.  The ice model input templates are
located in {\tt \$ICESRC}. These templates are fortran modules that contain
information on grid dimensions, number of ice thickness categories and
vertical resolution in the ice categories. The grid is determined in
the run script and the resolution {\tt \$RES} is set in {\bf csim.setup.csh}:
100x116 for the gx3v5 grid and 320x384 for gx1v3.  Depending on {\tt \$RES} and the
number of ice thickness categories {\tt \$NCAT} set in {\bf csim\_run}, the
appropriate input template {\bf ice\_model\_size.\$\{RES\}x\$\{NCAT\}} will
be copied into the directory where the ice model is being built and renamed
{\bf ice\_model\_size.F}.  Files exist for the following configurations:\\

\noindent{\bf \$CSIMDIR/input\_templates/ice\_model\_size.F.100x116x1 \\
\$CSIMDIR/input\_templates/ice\_model\_size.F.100x116x3 \\
\$CSIMDIR/input\_templates/ice\_model\_size.F.100x116x5 \\
\$CSIMDIR/input\_templates/ice\_model\_size.F.100x116x10 \\
\$CSIMDIR/input\_templates/ice\_model\_size.F.320x384x1 \\
\$CSIMDIR/input\_templates/ice\_model\_size.F.320x384x3 \\
\$CSIMDIR/input\_templates/ice\_model\_size.F.320x384x5 \\
\$CSIMDIR/input\_templates/ice\_model\_size.F.320x384x10} \\

\begin{Ventry}{NOTE:}
\item[NOTE]
Files exist only for certain numbers of ice thickness categories 
(1, 3, 5, and 10).  If you need a number of categories other than
these, the model will not run as is.  See Section \ref{ncat} for
information on how to change the number of ice thickness categories.
\end{Ventry}

The variable {\tt \$OML\_ICE\_SST\_INIT} is used if {\tt \$OCEANMIXED\_ICE} is
set to {\tt .true.} in the run script and determines the initial sea surface
temperature.  If the run is a startup run, this variable
is set to true in the ice setup script, and the January 1 value of the sea
surface temperature is read from the POP forcing file.  Thereafter, the
value of {\tt \$OML\_ICE\_SST\_INIT} is set to {\tt .false.}, and sea surface temperature
and the freeze/melt potential is read from a restart file.

\begin{table}
  \begin{center}
  \caption{Environment Variables Set in the Ice Setup Script}
  \label{table:environ_var}
  \begin{tabular}{p{4cm}p{10cm}} \hline

  Symbol              & Description    \\ \hline \hline
   {\tt HSTDIR} & directory where history files are written  \\
   {\tt RSTDIR} & directory where restart files are written  \\
   {\tt INIDIR} & directory where initial condition files are written  \\
   {\tt ICESRC} & directory where ice model input templates are located  \\
   {\tt RES}    & grid dimensions used to select model resolution  \\
   {\tt NLAT}   & number of latitudes in grid resolution \\
   {\tt NLON}   & number of longitudes in grid resolution \\
   {\tt OML\_ICE\_SST\_INIT}  & logical variable, if true initialize ocean mixed layer
                          temperature from within ice model\\
   {\tt RESTART} & logical variable used to initialize model from a restart file \\
   {\tt RSTFILE} & name of restart file \\
  \hline
  \end{tabular}
  \end{center}
\end{table}

The ice model contains two namelists.  The variables for both
lists are set in {\bf csim.setup.csh} and are written to the file {\bf ice\_in}
in {\tt \$EXEROOT} when the setup script is executed.  Changes to the
namelists must be made in the run or setup script, not in the {\bf ice\_in}
file.  The ice model reads his file at runtime.  Therefore, the namelist
will be updated even if the ice model is not recompiled.

One namelist is called {\tt icefields\_nml} and is defined in
{\bf  ice\_history.F}.  It contains a list of logical variables that
correspond to ice fields that will be written to the history file.  By
default, all these variables are set to {\tt .true.}, so leaving the namelist
blank will result in all fields being written to the history file.
The available fields are listed in Table \ref{table:history_fields}. Changing
the content of the history files via the namelist is discussed in section
\ref{change_content}.

The other namelist is called {\tt ice\_nml} and is defined in
{\bf ice\_init.F}. It contains variables that control the physics used in
the model.  They are listed in Tables \ref{table:nml_time}-\ref{table:nml_ocn}.
Some of the variables in the namelist are determined from environment variables
set in the scripts.  Variables that are commonly changed directly in the
namelist are the timestep {\tt dt}, the length of the model run {\tt npt}, and
the number of subcycles per timestep in the ice dynamics {\tt ndte}.

