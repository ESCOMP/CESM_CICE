%=======================================================================
% CVS: $Id: ice_nml_var.tex 5 2005-12-12 17:41:05Z mvr $
% CVS: $Source$
% CVS: $Name$
%=======================================================================
% Namelist Variables

CICE uses the same namelists for both the coupled and uncoupled models.
This section describes the namelist variables in the namelist {\tt ice\_nml},
which determine time management, output frequency, model physics, filenames,
and options for the mixed layer ocean model. 
The ice namelists for the coupled model are now located in
{\bf \$CASE/Buildconf}.  

A script reads the input namelist at runtime, and writes the namelist
information to the file {\bf ice\_in} in the directory where the model 
executable is located.  Therefore, the namelist will be updated even if the 
ice model is not recompiled.  The default values of {\tt ice\_nml} are set in 
{\bf ice\_init.F}.  If they are not set in the namelist in the script, they will
assume the default values listed in Tables \ref{table:nml_time}-\ref{table:nml_ocn},
which list all available namelist parameters.  The default values shown here are
for the coupled model, which is set up for a production run.  Several of the
varialbes have different values for the uncoupeld, which is set up for a 10
day test run with more frequent output.  Only a few of these variables are
required to be set in the namelist; these values are noted in the paragraphs
below.  An example of the default namelist is shown in Section \ref{example1_nml}.

\begin{table}[ht]
  \begin{center}
  \caption{Namelist Variables for Time Management}
  \label{table:nml_time}
   \begin{tabular}{p{2.0cm}p{2.0cm}p{4.0cm}p{6.5cm}} \hline
Varible  & Type & Default Value & Description               \\

\hline \hline

{\tt runid}   & Character &  'unknown\_runid' & Text used in netCDF
                                     global attributes \\

{\tt runtype} & Character &  'unknown\_runtype' & Determines if BASEDATE is
                          received from coupler or restart file   \\

{\tt istep0} &  Integer &   0   & Step counter, number of steps
                                     taken in previous integration \\

{\tt npt}    &  Integer & 99999 & Total number of timesteps in a model run,
                                    model stops when istep exceeds npt (not
                                    used in coupled runs)\\

{\tt dt}     &  Double  & 3600. & Timestep in seconds \\

{\tt ndyn\_dt}&  Integer  & 1 & Times to loop through ice dynamics \\

{\tt ndte}   &  Integer & 120   & Number of subcycles per timestep in
                                   ice dynamics \\

  \hline
  \end{tabular}
  \end{center}
\end{table}

The time management namelist options are shown in Table \ref{table:nml_time}.
{\tt runid} is a character string that contains descriptive information gathered
from the run script.  This information is written to the global attributes
in the history files.  {\tt runtype} is determined from the value of {\tt \$RUNTYPE}
set in the run script.  The options for this are discussed in section
\ref{runtypes}.  {\tt istep0} is the number of steps taken in a previous integration
and is written to the restart file. 

\subsection{Changing the Length of the Model Run}

The length of an uncoupled model run is controlled by the variable {\tt npt}.
It is the total number of time steps taken in an integration.  The value
of {\tt npt} is not used in a coupled run, since the point at which integration
is stopped is determined by the coupler.  The length of a coupled run should be
set in the 
\begin{htmlonly}
  \htmladdnormallink{CCSM scripts}{http://www.ccsm.ucar.edu/models/ccsm3.0/ccsm}.
\end{htmlonly}
\begin{htmlonly}
  CCSM scripts (http://www.ccsm.ucar.edu/models/ccsm3.0/ccsm).
\end{htmlonly}

\subsection{Changing the timestep}
\label{nml_time_mgmt}

{\tt dt} is the timestep in seconds for the ice model thermodynamics.
The thermodynamics component is stable but not necessarily accurate for any value
of the timestep.  The value chosen for {\tt dt} depends on the stability of the
transport and the grid resolution.  A conservative estimate of {\tt dt} for the
transport using the MPDATA advection scheme is

\begin{equation}
  \Delta t < \frac{min(\Delta x, \Delta y)}{4 max(u, v)} .
\end{equation}
Maximum values for {\tt dt} for the two standard CCSM POP grids, assuming 
$max(u,v) = 0.5 m/s$, are shown in Table \ref{table:max_dt}.  The default
timestep for CICE is one hour.

\begin{table}
  \begin{center}
  \caption{Maximum values for ice model timestep {\tt dt}}
  \label{table:max_dt}
  \begin{tabular}{lll} \hline
  Grid & $min(\Delta x, \Delta y)$ & $max \Delta t$   \\
  \hline \hline

gx3v5   & 28845.9 m & 4.0 hr \\
gx1v3   &  8558.2 m & 1.2 hr \\

  \hline
  \end{tabular}
  \end{center}
\end{table}

The calculation of the ice velocities is subcycled {\tt ndte} times per
timestep so that the elastic waves are damped before the next timestep. The
subcycling timestep is calculated as {\tt dte = dt/ndte}
and must be sufficiently smaller than the damping timescale {\tt T},
which needs to be sufficiently shorter than {\tt dt}
\begin{equation}
 dte < T < dt
\end{equation}
This relationship is discussed in \cite{hunk01}; also
see \cite{cice02}, section 4.4.  The best ratio for
[dte : T : dt] is [1 : 40 : 120]. Typical combinations of {\tt dt}
and {\tt ndte} are (3600., 120), (7200., 240) (10800., 120).

Occasionally, ice velocities are calculated that are larger than what is
assumed when the model timestep is chosen.  This causes a CFL violation
in the transport scheme.  A namelist option was added ({\tt ndyn\_dt})
to subcycle the dynamics to get through these instabilities that arise
during long integrations.  The default value for this variable is one,
and is typically increased to two when the ice model reaches an instability.
The value in the namelist should be returned to one by the user when the
model integrates past that point.

\begin{table}
  \begin{center}
  \caption{Namelist Variables for Writing Output}
  \label{table:nml_write}
   \begin{tabular}{p{2.0cm}p{2.0cm}p{2.0cm}p{6cm}} \hline
  Varible  & Type & Default  & Description               \\
\hline \hline
{\tt diagfreq} &  Integer &  24 & Frequency of diagnostics written
                     (min, max, hemispheric sums) to standard output   \\
         &          &     & 24  =$>$ writes once every 24 timesteps  \\
         &          &     & 1  =$>$ diagnostics written each timestep \\
         &          &     & 0  =$>$ no diagnostics written \\

{\tt histfreq} & Character & 'm' & Frequency of output written to 
                    history file \\
         &          &     & 'D' or 'd' writes daily data \\
         &          &     & 'W' or 'w' writes weekly data \\
         &          &     & 'M' or 'm' writes monthly data \\
         &          &     & 'Y' or 'y' writes yearly data \\
         &          &     & '1' writes every timestep \\
         &          &     & '0' no history data is written \\

{\tt hist\_avg}  &  Logical & .true. & If true, averaged history
                       information is written out at a frequency
                       determined by histfreq.  If false, instantaneous
                       values rather than time-averages are written. \\

{\tt dumpfreq} & Character & 'y' &  Frequency restart data is written to file \\
         &          &     & 'D' or 'd' writes restart every {\tt dumpfreq\_n} days \\
         &          &     & 'M' or 'm' writes restart every {\tt dumpfreq\_n} months \\
         &          &     & 'Y' or 'y' writes restart every {\tt dumpfreq\_n} years \\
         &          &     & '0' no restart data is written \\

{\tt dumpfreq\_n} & Integer & 1 & Frequency restart data is written to file \\

{\tt \scriptsize{print\_points}} & Logical & .false. & print diagnostic data for two grid points \\
  \hline
  \end{tabular}
  \end{center}
\end{table}

\subsection{Writing Output}

The namelist variables that control the frequency of the model diagnostics, netCDF
history files, and binary restart files are shown in Table \ref{table:nml_write}.
By default, diagnostics are written out once every 24 timesteps to the ascii file
{\bf ice.log.\$LID} (see section \ref{stdout}). {\tt \$LID} is a time stamp that is
set in the main script.   

{\tt histfreq} controls the output frequency of the 
netCDF history files; writing monthly averages is the default.  The content of the
history files is described in section \ref{history}.  The value of {\tt hist\_avg}
determines if instantaneous or averaged variables are written at the frequency set
by {\tt histfreq}.  If {\tt histfreq} is set to '1' for instantaneous output,
{\tt hist\_avg} is set to {\tt .false.} within the source code to avoid conflicts.
{\tt dumpfreq} and {\tt dumpfreq\_n} control the output frequency of the binary
restart files; writing one restart file per year is the default.

If {\tt print\_points} is {\tt .true.}, diagnostic data is printed out for two
grid points, one near the north pole and one near the Weddell Sea.  The points
are set at the top of {\bf ice\_diagnostics.F}.  This option can be helpful for
debugging.

\begin{table}
  \begin{center}
  \caption{Namelist Variables for Model Physics}
  \label{nml_physics}
  \begin{tabular}{p{3cm}p{2.0cm}p{3cm}p{6.5cm}} \hline
  Varible Name & Type & Default Value & Description               \\
\hline \hline

{\tt restart} & Logical & .false. & If true, model is initialized using a
   restart file, if false, model is initialized using initial conditions
   in {\bf ice\_init.F}. \\

{\tt kcolumn} &  Integer & 0 & Column model flag. \\
        &          &   & 0 = off \\
        &          &   & 1 = column model (not tested or supported)\\

{\tt kitd} &  Integer & 1 & Determines ITD conversion \\
     &          &   & 0 = delta scheme \\
     &          &   & 1 = linear remapping \\

{\tt kdyn} &  Integer & 1 & Determines ice dynamics \\
     &          &   & 0 = No ice dynamics\\
     &          &   & 1 = Elastic viscous plastic dynamics\\

{\tt kstrength} &  Integer & 1 &  Determines pressure formulation \\
          &          &   &  0 = \cite{hibl79} parameterization \\
          &          &   &  1 = \cite{roth75b} parameterization \\

{\tt evp\_damping} &  Logical & .false. & If true, use damping procedure
                                             in evp dynamics (not supported). \\

{\tt snow\_into\_ocn} &  Logical & .true. & If true, snow on ridged
                                                ice falls into ocean. \\

{\tt advection} &  Character & 'remap' &  Determines horizontal
                                               advection scheme. \\
          &            &   &  'remap' = incremental remapping \\
          &            &   &  'mpdata2' = second order advection \\
          &            &   &   'upwind' = first order advection \\

{\tt grid\_type} &  Character & 'displaced\_pole' &  Determines grid type. \\
          &            &   &  'displaced\_pole' or 'rectangular' (not supported) \\

{\tt albicev} &  Double & 0.73 &  Visible ice albedo \\

{\tt albicei} &  Double & 0.33 &  Near-infrared ice albedo \\

{\tt albsnowv} &  Double & 0.96 &  Visible snow albedo \\

{\tt albsnowi} &  Double & 0.68 &  Near-infrared snow albedo \\

{\tt no\_ice\_ic} & Logical & .false. &  Initializes ice model with no ice \\
 
  \hline
  \end{tabular}
  \end{center}
\end{table}

\subsection{Model Physics}

The namelist variables for the ice model physics are listed in Table 
\ref{nml_physics}.  {\tt restart} is almost always true since most
run types begin by reading in a binary restart file.  See section 
\ref{runtypes} for a description of the run types and about using
restart files and internally generated model data as initial conditions.
{\tt kcolumn} is a flag that will run the model as a single column if is
set to 1.  This option has not been thoroughly tested and is not supported. 

{\tt kitd} determines the scheme used to redistribute sea ice within the ice thickness
distribution (ITD) as the ice grows and melts.  The linear remapping scheme is the
default and approximates the thickness distribution in each category as a linear 
function (\cite{lips01}).  The delta function method represents {\it g(h)} in
each category as a delta function (\cite{bitz01}).  This method can leave some
categories mostly empty at any given time and cause jumps in the properties of
{\it g(h)}.

{\tt kdyn} determines the ice dynamics used in the model.  The default is the
elastic-viscous-plastic (EVP) dynamics \cite{hunk97}.  If {\tt kdyn} is set to 0,
the ice dynamics is inactive. In this case, ice velocities are not computed
and ice is not transported.  Since the initial ice velocities are read in
from the restart file, the maximum and minimum velocities written to the 
log file will be non-zero in this case, but they are not used in any calculations.

The value of {\tt kstrength} determines which formulation is used to
calculate the strength of the pack ice.  The \cite{hibl79} calculation
depends on mean ice thickness and open water fraction.  The calculation
of \cite{roth75b} is based on energetics and should not be used if the
ice that participates in ridging is not well resolved.  

{\tt evp\_damping} is used to control the damping of elastic waves in
the ice dynamics.  It is typically set to {\tt .true}. for high-resolution
simulations where the elastic waves are not sufficiently damped out in a
small timestep without a significant amount of subcycling.  This procedure
works by reducing the effective ice strength that's used by the dynamics
and is not a supported option.

The value of {\tt snow\_into\_ocean} determines what happens to the snow
on ice that is ridged.  The default value is {\tt .true.}, so the snow cover
on ice that undergoes ridging is put into the ocean.  If this variable is
{\tt .false.}, the snow on ice undergoing ridging remains on the ice.

{\tt advection} determines the horizontal transport scheme used. The default
scheme is the incremental remapping method (\cite{lipshunke04}).  This method
is less diffusive and is computationally efficient for large numbers of categories
or tracers.  The MPDATA scheme is also available.  It is second order accurate,
and more computationally expensive than remapping. The upwind scheme is only first
order accurate.
 
For coupled runs, both supported grids (gx3v5 and gx1v3) are {\tt 'displaced\_pole'}.
The 'rectangular' option for a regular grid with constant 
latitude and longitude spacing is not supported.

The values of the snow and ice albedos are now set in the namelist.  The ice albedos
are those for ice thicker than {\tt ahmax}, which is currently set at 0.5 m.  This
thickness is a parameter that can be changed in {\bf ice\_albedo.F}. The snow albedos
are for cold snow.  {\tt no\_ice\_ic} provides an option to initialize the ice model
with no ice cover.


\begin{table}
  \begin{center}
  \caption{Namelist Variables for File Names}
  \label{table:nml_file_info}
  \begin{tabular}{p{2.5cm}p{2.5cm}p{3cm}p{6.0cm}} \hline
  Varible & Type & Default Value & Description               \\
\hline \hline

{\tt grid\_file} &  Character & 'data.domain.grid' &  Input filename
                                           containing grid information. \\

{\tt kmt\_file} &  Character & 'data.domain.kmt' &  Input filename
                                       containing land mask information. \\

{\tt pointer\_file} & Character & 'ice.restart\_file' & Pointer file that
                                   contains the name of the restart file. \\

{\tt incond\_dir} & Character & ' ' & Directory where netCDF
           initial condition file is output. \\

{\tt restart\_dir} & Character & ' ' & Directory where restart 
           files are output. \\

{\tt history\_dir} & Character & ' ' & Directory where history 
           files are output. \\

{\tt incond\_file} & Character & 'incond' & Root name of netCDF
    output initial condition file. \\

{\tt dump\_file} &  Character & 'iced' &  Prefix for output file
                                             containing restart information. \\

{\tt history\_file} & Character & 'iceh' & Root name of history 
           files. \\
  \hline
  \end{tabular}
  \end{center}
\end{table}

\subsection{File Names}

The namelist parameters listed in Table \ref{table:nml_file_info} are for
initial condition, restart, and history file and directory information.
During execution, the ice model reads grid and land mask information from
the files {\tt grid\_file} and {\tt kmt\_file} that should be located in
the executable directory. There are commands in the scripts that copy these
files from the input data directory, rename them from 
{\bf global\_\${ICE\_GRID}.grid} and {\bf global\_\${ICE\_GRID}.kmt} to the
default filenames shown in Table \ref{table:nml_file_info}.

The namelist variable {\tt pointer\_file} is set to the name of the pointer file
containing the restart file name that will be read when model execution
begins.  The pointer file resides in the scripts directory and is created initially
by the ice setup script but is overwritten every time a new restart file is created.
It will contain the name of the latest restart file.  The default filename 
{\it ice.restart\_file} shown in Table \ref{table:nml_file_info} will not work
unless some modifications are made to the ice setup script and a file is created
with this name and contains the name of a valid restart file; this variable must
be set in the namelist.  More information on restart pointer files can be found
in section \ref{pointer_files}.

{\tt incond\_dir}, {\tt restart\_dir} and {\tt history\_dir} are the directories
where the initial condition file, the restart files and the history files will
be written, respectively.  These values are set at the top of the setup script
and have been modified from the default values to meet the requirements of the
CCSM filenaming convention.  This allows each type of output file to be written
to a separate directory.  If the default values are used, all of the output
files will be written to the executable directory.

{\tt incond\_file}, {\tt dump\_file} and {\tt history\_file} are the root
filenames for the initial condition file, the restart files and the history files,
respectively.  These strings have been determined by the requirements of the CCSM
filenaming convention, so the default values listed in Table \ref{table:nml_file_info}
are not the same as those shown in the namelist in Section \ref{example1_nml}.  See sections
\ref{restart_files} and \ref{history_files} for an explanation of how the rest
of the filename is created.

\begin{table}
  \begin{center}
  \caption{Namelist Variables for Ocean Mixed Layer Model}
  \label{table:nml_ocn}
  \begin{tabular}{p{4cm}p{2.0cm}p{3cm}p{5.5cm}} \hline
  Varible Name & Type & Default Value & Description               \\

  \hline \hline

{\tt oceanmixed\_ice} &  Logical  & .false.  &  If true, run model with
                                            ocean mixed layer model.  \\

{\tt oceanmixed\_ice\_file} &  Character & \scriptsize{'oceanmixed\_ice.nc'}  &  Name of
                                        file with ocean mixed layer data.  \\

{\tt \scriptsize{oceanmixed\_ice\_sst\_init}} &  Logical & .false.  &  If true, 
                             Jan 1 sst is read from forcing file.  \\

{\tt prntdiag\_oceanmixed} &  Logical & .false.  &  If true, print ocean
                                                mixed layer diagnostics.  \\

  \hline
  \end{tabular}
  \end{center}
\end{table}

\subsection{Ocean Mixed Layer Model}

An ocean mixed layer model has been incorporated into the ice model since
the CCSM data ocean component does not allow frazil ice growth to occur.
This is due to the minimum ocean mixed layer temperature being fixed at the
freezing point.  It is a simple slab ocean mixed layer model that is forced
using output from a POP ocean simulation.  More details on the physics of the
ocean mixed layer model can be found in the Physics Documentation.  This
option can be run with the gx3v5 or the gx1v3 grid. 

The namelist variables for the ocean mixed layer model within the ice model
are shown in Table \ref{table:nml_ocn}.  To use the slab ocean model,
{\tt \$OCEANMIXED\_ICE} must be set to {\tt .true.} in the namelist.
There are commands in the scripts that will copy the grid dependent forcing
file from the input data directory to the executable directory and rename it
{\bf oceanmixed\_ice.nc}.  This is generally not the best ocean forcing, but
can be used as a template for creating an ocean forcing file appropriate
for the application.

The variable {\tt \$OML\_ICE\_SST\_INIT} determines the initial sea surface
temperature.  For an initial or startup run, this variable should be
set to true, and the January 1 value of the sea surface temperature
will be read from the POP forcing file.  For continuation runs, the
value of {\tt \$OML\_ICE\_SST\_INIT} should be set to false, and sea
surface temperature and the freeze/melt potential will be read from a restart file.
This variable will be automatically set in the scripts depending on the
run type.  When the slab ocean mixed layer within the ice model is used,
the data that is received from the coupler from the ocean component
(docn or POP) is overwritten by the values calculated by the ocean mixed
layer. Therefore, it is not appropriate to use the ocean mixed layer option
coupled to an active ocean model.  Also, using the ice model with the slab ocean
mixed layer turned on, coupled to an active atmosphere and a data ocean model
will require changes to the coupler, since the ocean values calculated in the
ice model will not be sent to the coupler and received by the atmosphere component.
